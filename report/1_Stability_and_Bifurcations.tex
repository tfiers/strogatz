% !tex root = ./NLS_Report.tex


\chapter{Stability of equilibrium points \& bifurcations}




\section{Simple population model}

The population model has in general two solutions (and hence two fixed points) for $\dot{N} = 0$ , namely
\[
N_1 = 0 \qq{and} N_2 = K \frac{\alpha - \beta}{\alpha}.
\]
The stability of these fixed points in function of $\alpha$ and $\beta$ can be summarised as follows:

\begin{center}
\begin{tabular}{@{}ll@{}} \toprule
Parameter region    &  Fixed points \\ \midrule
\multirow{2}{*}
{$\alpha < \beta$}  &  $N_1 = 0$: stable \\
                    &  $N_2 < 0$: unstable \\[1em]
\multirow{2}{*}
{$\alpha = \beta$}  &  $N_1 = N_2 = 0$: half-stable \\
                    &  (unstable for $N < 0$, stable for $N > 0$)  \\[1em]
\multirow{2}{*}
{$\alpha > \beta$}  &  $N_1 = 0$: unstable \\
                    &  $N_2 > 0$: stable \\
\bottomrule
\end{tabular}
\end{center}

The system thus undergoes a transcritical bifurcation at $\alpha = \beta$. Note that the fixed point $N_2 < 0$ is not meaningful in this model, as $N$ represents a non-negative population count.

For the given parameter values, $\alpha > \beta$. Using the above results, we therefore find an unstable fixed point $N_1 = 0$, and a stable fixed point $N_2 = K (\alpha - \beta)/\alpha = \num{4 023 913}$. As the population starts at $N > 0$, it will evolve towards $N_2$. The difference between $N(t)$ and $N(\infty) = N_2$ decays exponentially, as a Taylor approximation of $\dot{N}$ around $N_2$ can show.




\section{Gene control model}

For $r = 0$, the system equations become decoupled:
%
\begin{align*}
\dot{x} &= \frac{\alpha_1}{2} - x  \\
\dot{y} &= \frac{\alpha_2}{2} - y.
\end{align*}
%
We can therefore analyse them separately. It is clear that there is one fixed point, at $x^* = \alpha_1/2$ and $y^* = \alpha_2/2$. It is a globally stable attractor, as $\forall x < x^*,\ \dot{x} > 0$ and $\forall x > x^*,\ \dot{x} < 0$ (and analogously for $\dot{y}$). The fixed point is thus an attracting star.

For $r \geq 0$ and $\alpha_1 = \alpha_2 = 2$, the equilibrium equations become
%
\begin{align*}
x (1+y^r) &= 2 \\
y (1+x^r) &= 2.
\end{align*}
%
It can be easily verified that $(1,1)$ is a solution and hence a fixed point. We have already shown that it is the only fixed point for $r = 0$. Plotting the gradient in phase space for different $r \in (0, 2)$ strongly suggests that it is also the only fixed point for nonzero $r$ (at least for $x \geq 0$ and $y \geq 0$). See e.g. \cref{fig:pplane_r1} for $r = 1$. As a final piece of evidence, different trajectories simulated back in time all either tend towards $(\infty, \infty)$ or cross into $x < 0,\ y < 0$ territory.

\begin{figure}
\img[0.4]{1_2_streamline_close}
\img[0.4]{1_2_quiver_close}
\img[0.4]{1_2_quiver_far}
\captionn{There is only one fixed point for $\bm{0 < r < 2}$}{Phase space plots of the gene expression model, for $r = 1$ and $\alpha_1 = \alpha_2 = 2$. \emph{Left}: some (partial) trajectories in phase space. Thicker lines represent a higher local speed. Note the attractor at $(1, 1)$. \emph{Middle}: local velocities, evaluated on a grid. \emph{Right}: same as middle, but for a larger region in phase space.}
\label{fig:pplane_r1}
\end{figure}
