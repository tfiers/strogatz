% !tex root = ./NLS_Report.tex


\chapter{Stability of equilibrium points \& bifurcations}



\section{A simple population model}

The population model has in general two solutions (and hence two fixed points) for $\dot{N} = 0$ , namely
\[
N_1 = 0 \qq{and} N_2 = K \frac{\alpha - \beta}{\alpha}.
\]
The stability of these fixed points in function of $\alpha$ and $\beta$ can be summarised as follows:

\begin{center}
\begin{tabular}{@{}ll@{}} \toprule
Parameter region    &  Fixed points \\ \midrule
\multirow{2}{*}
{$\alpha < \beta$}  &  $N_1 = 0$: stable \\
                    &  $N_2 < 0$: unstable \\[1em]
\multirow{2}{*}
{$\alpha = \beta$}  &  $N_1 = N_2 = 0$: half-stable \\
                    &  (unstable for $N < 0$, stable for $N > 0$)  \\[1em]
\multirow{2}{*}
{$\alpha > \beta$}  &  $N_1 = 0$: unstable \\
                    &  $N_2 > 0$: stable \\
\bottomrule
\end{tabular}
\end{center}

The system thus undergoes a transcritical bifurcation at $\alpha = \beta$. Note that the fixed point $N_2 < 0$ is not meaningful in this model, as $N$ represents a non-negative population count.

For the given parameter values, $\alpha > \beta$. Using the above results, we therefore find an unstable fixed point $N_1 = 0$, and a stable fixed point $N_2 = K (\alpha - \beta)/\alpha = \num{4 023 913}$. As the population does not start at $N = 0$, it will evolve towards $N_2$. The difference between $N(t)$ and $N(\infty) = N_2$ decays exponentially, as a Taylor approximation of $\dot{N}$ around $N_2$ can show.



\section{Gene control model}


