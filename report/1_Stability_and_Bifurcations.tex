% !tex root = ./NLS_Report.tex


\chapter{Stability of equilibrium points \& bifurcations}




\section{Simple population model}

The population model has in general two solutions (and hence two fixed points) for $\dot{N} = 0$ , namely
\[
N_1 = 0 \qq{and} N_2 = K \frac{\alpha - \beta}{\alpha}.
\]
The stability of these fixed points in function of $\alpha$ and $\beta$ can be summarised as follows:

\begin{center}
\begin{tabular}{@{}ll@{}} \toprule
Parameter region    &  Fixed points \\ \midrule
\multirow{2}{*}
{$\alpha < \beta$}  &  $N_1 = 0$: stable \\
                    &  $N_2 < 0$: unstable \\[1em]
\multirow{2}{*}
{$\alpha = \beta$}  &  $N_1 = N_2 = 0$: half-stable \\
                    &  (unstable for $N < 0$, stable for $N > 0$)  \\[1em]
\multirow{2}{*}
{$\alpha > \beta$}  &  $N_1 = 0$: unstable \\
                    &  $N_2 > 0$: stable \\
\bottomrule
\end{tabular}
\end{center}

The system thus undergoes a transcritical bifurcation at $\alpha = \beta$. Note that the fixed point $N_2 < 0$ is not meaningful in this model, as $N$ represents a non-negative population count.

For the given parameter values, $\alpha > \beta$. Using the above results, we therefore find an unstable fixed point $N_1 = 0$, and a stable fixed point $N_2 = K (\alpha - \beta)/\alpha = \num{4 023 913}$. As the population starts at $N > 0$, it will evolve towards $N_2$. The difference between $N(t)$ and $N(\infty) = N_2$ decays exponentially, as a Taylor approximation of $\dot{N}$ around $N_2$ can show.




\section{Gene control model}

For $r = 0$, the system equations become decoupled:
%
\begin{align*}
\dot{x} &= \frac{\alpha_1}{2} - x  \\
\dot{y} &= \frac{\alpha_2}{2} - y.
\end{align*}
%
We can therefore analyse them separately. It is clear that there is one fixed point, at $x^* = \alpha_1/2$ and $y^* = \alpha_2/2$. It is a globally stable attractor, as $\forall x < x^*,\ \dot{x} > 0$ and $\forall x > x^*,\ \dot{x} < 0$ (and analogously for $\dot{y}$ and $y^*$). The fixed point is thus an attracting star.

For $r \geq 0$ and $\alpha_1 = \alpha_2 = 2$, the equilibrium equations become
%
\begin{align*}
x (1+y^r) &= 2 \\
y (1+x^r) &= 2.
\end{align*}
%
It is easily verified that $(1,1)$ is a solution and hence a fixed point. We have already shown that it is the only fixed point for $r = 0$. Plotting the gradient in phase space for different $r \in (0, 2)$ strongly suggests that it is also the only fixed point for nonzero $r < 2$ (at least for $x \geq 0$ and $y \geq 0$). See e.g. \cref{fig:genes_r1} for $r = 1$. As a final piece of evidence, different trajectories simulated back in time all either tend towards $(\infty, \infty)$ or cross into forbidden $x < 0,\ y < 0$ territory.

\begin{figure}
\img[0.4]{1_2_streamline_close}
\img[0.4]{1_2_quiver_close}
\img[0.4]{1_2_quiver_far}
\captionn{There is only one fixed point for $\bm{0 \leq r \leq 2}$}{Phase space plots of the gene control model, for $r = 1$ and $\alpha_1 = \alpha_2 = 2$. \emph{Left}: some (partial) trajectories in phase space. Thicker lines represent a higher local speed. Note the attractor at $(1, 1)$. \emph{Middle}: local velocities, evaluated on a grid. \emph{Right}: same as middle, but for a larger region in phase space.}
\label{fig:genes_r1}
\end{figure}

% To analyse the stability of the $(1,1)$ fixed point, we study the behaviour of a small perturbation $\vb{u} = (x - 1, y - 1)$. To a first order approximation (Chapter 6 in Strogatz \cite{Strogatz1994}), this point will move through phase space according to:
% %
% \begin{equation}\label{eq:linear}
% \vb{\dot{u}} = A \ \vb{u},
% \end{equation}
% %
% where $A$ is the Jacobian of $(\dot{x}, \dot{y})$, evaluated in $(1, 1)$. The solution of \cref{eq:linear} is $\vb{u}(t) = c_1 e^{\lambda_1 t} \vb{v}_1 + c_2 e^{\lambda_2 t} \vb{v}_2$, with $(\lambda_1, \vb{v}_1)$ and $(\lambda_2, \vb{v}_2)$ the eigenvalues and eigenvectors of $A$.

To analyse the stability of the $(1,1)$ fixed point, we approximate $(\dot{x}, \dot{y})$ as a linear system around this point. The Jacobian of $(\dot{x}, \dot{y})$ evaluated in $(1,1)$ is:
\[
\begin{pmatrix}
-1 & -\frac{r}{2}  \\
-\frac{r}{2} & 1
\end{pmatrix}.
\]
It has two distinct eigenvalue-eigenvector pairs:
%
\begin{align*}
\lambda_1 &= -\frac{r}{2} - 1   \qc \vb{v}_1 = (1,\ 1)  \\
\lambda_2 &= \frac{r}{2} - 1    \qc \vb{v}_2 = (-1,\ 1).
\end{align*}
%
For $0 \leq r < 2$, $\ \lambda_1 \in (-2, -1]$ and $\lambda_2 \in [-1, 0)$. Both eigenvalues are negative, and $(1,1)$ is therefore a stable node. Because $\lambda_1 < \lambda_2$, $\ \vb{v}_1 = (1,\ 1)$ is the fast eigendirection and $\vb{v}_2 = (-1,\ 1)$ is the slow eigendirection, as can be seen in \cref{fig:genes_r1}. (For $r = 0$, both eigenvalues are equal; i.e. $(1,1)$ is then a star node).


Based on phase space plots for different values of $r$, it seems that a supercritical pitchfork bifurcation occurs at $r = 2$. This is a plausible bifurcation given the system that is modelled: when the mutual repression rate $r$ is high, a slight abundance of one gene (say $x$) over the other will create a positive feedback loop: $x$ represses $y$ more than $y$ represses $x$, resulting in more $x$ and less $y$. This again results in even more $x$ and even less $y$, etcetera. Vice versa for an initial slight abundance of $y$. There is thus a precarious balance when $x = y$ (the unstable fixed point at $(1,1)$ for $r > 2$). On the other hand, when the mutual repression rate $r$ is low, both genes can be expressed in equal amounts. (For large $r$, this system might be called a winner-takes-all model, or a competitive model).

\Cref{fig:gene_phases} shows the different types of phase portrait that occur as $r \ge 0$ is varied. \Cref{fig:gene_bifurc} sketches the bifurcation diagram for this gene expression model.

\begin{figure}
\img[0.4]{1_2_star}
\img[0.4]{1_2_at_the_fork}
\img[0.4]{1_2_WTA}
\label{fig:gene_phases}
\end{figure}

\begin{figure}
\img[0.7]{1_2_bifurc_sketch}
\label{fig:gene_bifurc}
\end{figure}
