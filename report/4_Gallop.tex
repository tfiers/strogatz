% !tex root = ./NLS_Report.tex
\graphicspath{{../figures/4/}}


\chapter{Aero-elastic galloping}

From Newton's second law and the definitions of linear acceleration $\ddot{y}$ and speed $\dot{y}$, the motion of the bridge element is described by
%
\begin{align*}
\dot{x} &= - \qty(\rho\ V^2\ a\ C(x/V)\ /2  +  r x  +  k y) / m \\
\dot{y} &= x
\end{align*}
%
with $m, \rho, r, k, a, $ and $C(\alpha)$ as given.


\Cref{fig:eigencup} shows the eigenvalues of Jacobian at the fixed point (at the origin) as a function of $V$. At $V = V_C = 809.37...$ the eigenvalues land on the real axis, and the system undergoes a bifurcation. $V_C$ was determined numerically as the solution to $4 \Delta = \tau^2$ (see \cref{sec:predator}). Additionally, the trace of the Jacobian is $\tau = -0.023475.. V − 1$. For every $V \geq 0$, $\tau < 0$. Hence, the fixed point is locally never an unstable spiral or node, but rather a stable spiral or node. Based on this linear analysis, we can classify the stability of the fixed point as in \cref{tab:fp_class}.

\begin{figure}
\img[0.8]{eigencup}
\captionn{Linear stability analysis of the bridge model}{Eigenvalues of the Jacobian at the fixed point $(0,0)$, for $V \in [0, 1000]$ ($V = 0$ at the ends of the bends, near the imaginary axis).}
\label{fig:eigencup}
\end{figure}

\begin{table}
\begin{tabular}{@{}ll@{}} \toprule
Wind speed $V$  & Stability of origin \\\midrule
$V < V_C$  &      Stable spiral  \\[1em]
$V = 0$    &      Stable star  \\[1em]
$V > V_C$  &      Stable node  \\[1em]
\bottomrule
\end{tabular}
\captionn{Classification of the fixed point}{}
\label{tab:fp_class}
\end{table}


\begin{figure}
\img[1]{bridge_phase}
\captionn{Phase space diagram of the bridge model}{Here for $V = 300$.}
\label{fig:bridge_phase}
\end{figure}
