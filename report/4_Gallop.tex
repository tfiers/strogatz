% !tex root = ./NLS_Report.tex
\graphicspath{{../figures/4/}}


\chapter{Aero-elastic galloping}

Applying Newton's second law to the bridge element, and introducing $x = \dot{y}$ as its linear velocity, its dynamic behaviour can be described by
%
\begin{align*}
\dot{x} &= 0.5\ V^2\ C(x/V)\ - x - 100y \\
\dot{y} &= x
\end{align*}
%
where the constants have been substituted in, and with $C(\alpha) \approx \num{e-2}\ \alpha - \num{e-3}\ \alpha^3 + \num{e-5}\ \alpha^5 - \num{e-8}\ \alpha^7$ (exact coefficients as in the assignment).


The system has a fixed point at the origin. \Cref{fig:eigencup} shows the eigenvalues of the Jacobian at this fixed point, for varying wind speeds $V$. In low-wind conditions, the system starts out stable, with both eigenvalues having negative real parts (but having nonzero imaginary parts, predicting oscillatory behaviour). At $V = V_C = 42.5985...$, the eigenvalues cross the imaginary axis, and the origin transitions from a stable spiral, through a center, to an unstable spiral. This is a telltale sign of a Hopf bifurcation. $V_C$ is the solution to $\tau = 0$, with $\tau = V\ A_1 / 2 - 1$ the trace of the Jacobian at the origin. For very large $V = V_R = 894.6..$, the eigenvalues land on the real axis, and the unstable spiral transitions to a repelling star and a repelling node. $V_R$ is the solution to $4 \Delta = \tau^2$, with $\Delta = 100$ the determinant of the Jacobian at the origin (see \cref{sec:predator}). \Cref{tab:fp_class} summarises this classification of the origin's stability.

\begin{figure}
\begin{minipage}[b]{0.6\textwidth}
\img[1]{eigencup}
\captionn{Linear stability analysis of the bridge model}{Eigenvalues of the Jacobian at the fixed point $(0,0)$, for $V \in [0, 900]$ ($V = 0$ in the left half plane, at $\text{Re } \lambda_i = -0.5$).}
\label{fig:eigencup}
\end{minipage}
\hfill
\begin{minipage}[b]{0.6\textwidth}
\centering
\begin{tabular}{@{}ll@{}} \toprule
Wind speed $V$    & Stability of origin \\
\midrule
$V < V_C$         & Stable spiral       \\[1em]
$V = V_C$         & Center              \\[1em]
$V_C < V < V_R $  & Unstable spiral     \\[1em]
$V = V_R$         & Repelling star      \\[1em]
$V > V_R$         & Repelling node      \\[1em]
\bottomrule
\vspace{36pt}
\end{tabular}
\captionnoftable{Classification of the fixed point}{According to the linear stability analysis. $V_C \approx 42.6$ and $V_R \approx 895$.}
\label{tab:fp_class}
\vspace{14pt}
\end{minipage}
\end{figure}

When the nonlinear terms of $C(\alpha)$ are not neglected, $\dot{x}(x)$ will look `bumpy'. A slight displacement $y$ will shift this bumpy function, so that $\dot{x}(x)$ crosses the $x$-axis more than once. In one-dimensional systems, this tends to result in fixed points away from the origin. This might then correspond to one or more limit cycles in the current two-dimensional system. When the fixed point starts at the origin, we get an ordinary supercritical Hopf bifurcation. When $V$ then passes the bifurcation $V_C$, the origin is unstable, and even the tiniest displacement from the origin will end up on the limit cycle. This is indeed what we observe when the model is simulated (\cref{fig:bridge_phase}).

When the `bump' crosses the $x$-axis away from the origin, we get a limit cycle `out of the blue sky', i.e. a saddle-node bifurcation of a cycle. This is indeed what happens and what we can observe in a bifurcation diagram (\cref{fig:bd_coco}): at $V \approx 1.24 V_C$, an unstable-stable pair of limit cycles appears. Another saddle-node bifurcation occurs at $V \approx 1.84 V_C$: here, the two smallest limit cycles coalesce and annihilate. A trajectory on the smallest stable orbit will suddenly jump to the only orbit, with a catastrophically large amplitude. 

\begin{figure}
\img[0.6]{bridge_phase_subc}
\img[0.6]{bridge_phase_superc}
\captionn{Phase space diagrams of the bridge model}{\Left: below the critical wind speed $V_C$, displacements return to the origin. \Right: above the critical wind speed, the same starting phase point will end up on a limit cycle.}
\label{fig:bridge_phase}
\end{figure}


\begin{figure}
\img[0.75]{coco}
\captionn{Aero-elastic galloping bifurcation diagram}{$A$ is the maximal displacement during the limit cycle. Blue and green dots indicate stable fixed points and stable limit cycles, respectively. Red dots indicate unstable fixed points and unstable limit cycles. Black circles mark saddle-node bifurcations.}
\label{fig:bd_coco}
\end{figure}


\begin{figure}
\img[0.75]{orbits_3D}
\captionn{Phase space orbits}{$(y, \dot{y})$ traces of fixed points and limit cycles for different wind speeds $V$. Note the inner manifold, indicative of hysteresis behaviour.}
\label{fig:orbits_3D}
\end{figure}


The continuations of \cref{fig:bd_coco,fig:orbits_3D} were performed with the following settings:
\begin{itemize}
\item 12 discretisation intervals \texttt{NTST} for the collocation algorithm.
\item 4 collocation points \texttt{NCOL} per discretisation interval.
\item Adaptation of orbit discretisation between continuation steps (\texttt{NAdapt = 1})
\item Maximum parameter change \texttt{h\_max} of 100 per step.
\item Initial parameter change \texttt{h0} of 80 per step.
\item Maximum of \texttt{ItMX = 400} continuation steps per branch.
\item Sufficient accuracy of root finding \texttt{corr.TOL} of \num{1e-4}.
\end{itemize}

The numeric experiment detailed in \cref{tab:numerics} reveals that these parameters are important to calculate at least the cycle period: adding only one collocation point more per interval (or doubling the number of intervals) increases the accuracy of the calculated period by four orders of magnitude.


\begin{table}
\begin{tabular}{@{}rrrr@{}} \toprule
              &  \multicolumn{3}{c}{\texttt{NTST}}     \\ \cmidrule{2-4}
\texttt{NCOL} & 12              & 24              & 48             \\[1em]
\midrule
4             & \num{-1.0e-4}   & \num{1.1e-8}   & \num{1.2e-8}    \\[1em]
5             & \num{1.1e-8}    & \num{7.4e-9}   & \num{2.6e-9}    \\[1em]
6             & \num{2.4e-9}    & \num{4.7e-9}   & \num{3.0e-9}    \\[1em]
7             & \num{1.2e-8}    & \num{2.6e-9}   & 0               \\
% With nadapt = 1:
% 4             & \num{+9.6e-9}   & \num{+8.8e-9}   & \num{+8.0e-9}   \\[1em]
% 5             & \num{+4.5e-12}  & \num{-2.2e-12}  & \num{-3.9e-12}  \\[1em]
% 7             & \num{-2.3e-12}  & \num{-1.2e-12}  & 0               \\
\bottomrule
\end{tabular}
\captionn{Relative period error}{Period of the limit cycle at $V = 1.5 V_C$: relative error with respect to the period at number of intervals \texttt{NTST = 24} and number of collocation points \texttt{NCOL = 7}. Periods calculated without adaptive orbit discretisation (\texttt{NAdapt = 0}).}
\label{tab:numerics}
\end{table}


% p = [ 0.628319662973291, 0.628384555144029, 0.628384551817114;
%       0.628384555839797, 0.628384553490447, 0.628384550448574;
%       0.628384556729517, 0.628384550462603, 0.628384548843046 ]

% (p - p(end)) / p(end)

% 1.0e-03 *

%   -0.103258219627479   0.000010027272440   0.000004732878931
%    0.000011134505265   0.000007395791364   0.000002555008787
%    0.000012550389799   0.000002577334288                   0

% p6 = [ 0.628384550350641, 0.628384551814621, 0.628384550701407 ]
% pend = 0.628384548843046
% (p6 - pend) / pend
