% !tex root = ./NLS_Report.tex
\graphicspath{{../figures/5/}}


\chapter{Chaos}



\section{Lyapunov exponents of the Lorenz equations}
\label{sec:lorenz}

The algorithm integrates both an initial phase point and the spatial gradient of the phase point along the trajectory over fixed time steps \texttt{st} (via the simplest first order Euler method). Next, an orthogonal basis is sought for the new spatial gradient, via the Gram-Schmidt algorithm. The cumulative sum of logarithms of the Gram-Schmidt scaling factors divided by the total time elapsed is then the estimate of the Lyapunov exponent. This is integration-orthogonalisation loop is repeated for a given number of iterations \texttt{kkmax}.

When the time step is too large (e.g. \texttt{st = 0.1} here), the exponents diverge. When it is too small on the other hand (\texttt{st = 0.001} here), the convergence is extremely slow. Even for a balanced time step (like \texttt{st = 0.01} here), enough iterations need to be taken yield a decent result (e.g. \texttt{kkmax > 400} here; see \cref{fig:LE_Lorenz}). The initial phase point also influences the results. A different phase point as in \cref{fig:LE_Lorenz} ($(6,6,6)$) yields better estimates for example ($0.89, -0.04, $ and $-14$).

Interestingly, using a more advanced ODE solver (like a higher order Runge-Kutta method) or orthogonalisation algorithm (like a singular value decompostion via the QR-algorithm) yields markably worse results.


\begin{figure}
\img[0.8]{LE_Lorenz}
\captionn{Estimating Lyapunov exponents of the Lorenz system}{Initial phase point $(1,1,1)$. Final exponent estimates are $0.79, -0.07, $ and $-14.7$.}
\label{fig:LE_Lorenz}
\end{figure}






\section{Hindmarsh-Rose neuron model}

The system behaves chaotically in random burst mode (\cref{fig:neuron}, top): even the tiniest perturbation to the initial phase point renders $x(t)$eventually unpredictable. The burst generation mode is not chaotic: even relatively large perturbations do not qualitatively change the phase space trajectories (\cref{fig:neuron}, bottom).

\begin{figure}
\img[1]{random_burst_structure}\\[3em]
\img[1]{burst_generation}\\[0.5em]
\captionn{Hindmarsh-Rose neuron model}{See text for interpretation. (In random burst mode, the initial slow current was $z(0) = 3$, while in burst generation mode, it was $z(0) = 0$. Other parameters as instructed).}
\label{fig:neuron}
\end{figure}




\section{Chua's circuit}

A simulation of the Chua circuit displays chaotic behaviour (\cref{fig:chua}). Estimating the Lyapunov exponents according to the method of \cref{sec:lorenz} confirms this: the largest exponent $\approx 0.16 > 0$ (\cref{fig:LE_chua}).


\begin{figure}
\img[1.25]{chua}
\captionn{Chua circuit model}{Time series (top) and phase space projections (bottom). Note the chaotic behaviour and the double-scroll dynamics. (Initial condition $(1, y_0, 0)$, with $y_0$ as indicated).}
\label{fig:chua}
\end{figure}

\begin{figure}
\img[0.6]{LE_chua_A}
\img[0.6]{LE_chua_B}
\captionn{Estimating Lyapunov exponents of the Chua circuit}{\Left: \texttt{st = 0.01}, final largest exponent $0.146$. \Right: \texttt{st = 0.005}, final largest exponent $0.184$.}
\label{fig:LE_chua}
\end{figure}
