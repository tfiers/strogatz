% !tex root = ./NLS_Report.tex
\graphicspath{{../figures/3/}}


\chapter{Study of a predator-prey model}



We study the system
%
\begin{align*}
\dot{x} &= x (x-a) (1-x) - bxy  \\
\dot{y} &= xy - cy - d,
\end{align*}
%
with $a = 0.1$ and $b = 1.5$. The following could be an ecological interpretation of this system as a predator-prey model:

The $-d$ term represents a constant decline of species $y$; This would correspond to a linear decrease in $y$ over time, with slope $d$, if this was the only term present. Maybe an environmental agency eliminates a fixed number $d$ of $y$-type animals every time period, to keep the ecosystem balanced.

The $-cy$ term represents a proportional pressure on $y$, corresponding to an exponential decline of $y$ over time with time constant $1/c$ (i.e. faster decline for larger $c$). There might be a fixed amount of resources available for the $y$ species. Then, a larger number of $y$ animals will result in a proportionally smaller amount of resources per animal.

The $xy$ term represents a growth of $y$ that is both proportional to the other species and to itself. For constant $x$, this would correspond to exponential growth of $y$ with time constant $1/x$ (i.e. faster growth for more $x$). $y$ could be a multiplying parasite, and $x$ could be its host.

The $-bxy$ term represents a decline of the $x$ species proportional to both itself and to the other species $y$. For constant $y$, this would correspond to an exponential decline of $x$ with time constant $1/(by)$. The $y$ parasite might be pathological for $x$. Both more parasites $y$ and more hosts $x$ yield a higher probability of transmitting the parasite between hosts.

Finally, the $x (1-x)$ factor of the first term describes logistic growth (i.e. exponential growth from the origin, which switches halfway to exponential decay up to a carrying capacity -- which is $1$ in this case). This is a common model for constrained species growth. The $(x-a)$ multiplier has the effect that the growth does not start until $x$ reaches $a$: for $x < a$, the species will decline instead of grow. This could model the fact that more than a few individuals are necessary for succesful long-term reproduction.




\section{A qualitative study for $d = 0$}
% 0,1  1,5     0,38  0,48

Simulating the system for $y = 0$ confirms the predictions made above for the standalone behaviour of $x(t)$ (\cref{fig:time_sim}): logistic growth above the threshold $a$, and decay to zero below this threshold.

\begin{figure}
\img[1]{time_sim}
\captionn{Behaviour of $\bm{x}$ without $\bm{y}$}{Simulated trajectories $x(t)$ for $y = 0$ and different initial values $x_0$ (from top to bottom: \numlist{1.5; 1.1; 1.0; 0.9; 0.5; 0.13; 0.1; 0.07; 0}). Note the stable fixed points at $0$ and at the carrying capacity $1$, and the unstable fixed point at $a = 0.1$.}
\label{fig:time_sim}
\end{figure}

The system has four fixed points for $d = 0$. Three of these lie on the $x$-axis. They are listed in \cref{tab:fpx}, together with the eigenvalues and corresponding eigenvectors of the Jacobian in these points. The fourth fixed points has coordinates $(c,\ (c-a)(1-c)/b)$, and the eigenstructure of its Jacobian is rather more.. complex. Its trace $\tau$ and determinant $\Delta$ have simpler analytical expressions however: $\tau = c(1+a-2c)$ and $\Delta = c(c-a)(1-c)$. The following paragraphs describe the topological structure near these four fixed points.

\begin{table}
\begin{tabular}{@{}llll@{}}
\toprule
Fixed point  & Eigenvalues  & Eigenvectors \\
\midrule
\multirow{2}{*}
{$(0,\ 0)$}  & $-a$         & $(1,\ 0)$             \\[0.2em]
             & $-c$         & $(0,\ 1)$             \\[1.2em]
\multirow{2}{*}
{$(a,\ 0)$}  & $a-a^2$      & $(1,\ 0)$             \\[0.2em]
             & $a-c$        & $(ab/(c-a^2),\ 1)$    \\[1.2em]
\multirow{2}{*}
{$(1,\ 0)$}  & $a-1$        & $(1,\ 0)$             \\[0.2em]
             & $1-c$        & $(-b/(2-c-a),\ 1)$     \\
\bottomrule
\end{tabular}
\captionn{Fixed points on the $\bm{x}$-axis}{$(x,y)$-coordinates of the fixed points, and eigenvalue-eigenvector pairs of the Jacobian.}
\label{tab:fpx}
\end{table}

$(0,\ 0)$ is an attractor node. In the common case that $c > a = 0.1$, the $x$-axis is the slow eigendirection. When $c < a$, the $y$-axis is the slow eigendirection.

$(a,\ 0)$ is a saddle when $c > a$. Its stable manifold is then the $x$-axis, and its unstable manifold is locally spanned by $(ab/(c-a^2),\ 1)$, which is an upwards pointing vector, rotated slightly right. When $c < a$, both dimensions are unstable, and $(a,\ 0)$ is then a repellor node. For $c > a^2$, the slow eigendirection is $(ab/(c-a^2),\ 1)$. For $c < a^2$, the $x$-axis is the slow eigendirection.

$(1,\ 0)$ is a saddle when $c < 1$. With $a = 0.1$, the stable manifold is the $x$-axis, and the unstable manifold is locally spanned by $(-b/(2-c-a),\ 1)$, which is an upwards pointing vector, rotated slightly \emph{left}. When $c > 1$, both dimensions become stable, and $(1,\ 0)$ is then an attracting node. Because $c < 1.5$ in this exercise, the slow eigendirection is $(-b/(2-c-a),\ 1)$.

The final fixed point, $(c,\ (c-a)(1-c)/b)$, has a wider range of behaviours (\cref{fig:eigenheart}). It is a saddle for $c > 1$, and for $0 < c < a$. It is an unstable node for $a < c < c_u$, an unstable spiral for $c_u < c < (a+1)/2$, a stable spiral for $(a+1)/2 < c < c_s$, and a stable node for $c_s < c < 1$. For $a = 0.1$, $c_u$ and $c_s$ are numerically determined to be $c_u = 0.1259..$ and $c_s = 0.8783..$.

\begin{figure}
\img[0.55]{eigenheart}
\img[0.65]{stability_diagram}
\captionn{Linear system analysis of the fixed point $\bm{(c,\  (a-c)(c-1)/b)}$}{\Left: eigenvalues of the Jacobian at the fixed point, for $a = 0.1$ and $c \in [0, 1.5]$ ($c = 1.5$ at the far right, and at the far left outside the figure). \Right: trace $\tau$ and determinant $\Delta$ of the Jacobian at the fixed point, for $a = 0.1$ and $c \in [0, 1.04]$ (for larger $c$, the $(\tau, \Delta)$ curve simply extends further down in the 3rd quadrant). u.n.: unstable nodes.}
\label{fig:eigenheart}
\end{figure}

The separatrices between basins of attraction are the stable manifolds of saddles.

The stable manifold of a saddle $(x^*,\ y^*)$ can be found numerically by simulating a trajectory back in time. The initial point of this trajectory is $(x^* + \Delta x,\ y^* + \Delta y)$, where $\Delta x$ and $\Delta y$ are small perturbations from the fixed point on the line spanned by the eigenvector of the Jacobian corresponding to this manifold (i.e. the eigenvector with a positive eigenvalue).

