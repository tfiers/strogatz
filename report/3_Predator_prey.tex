% !tex root = ./NLS_Report.tex
\graphicspath{{../figures/3/}}


\chapter{Study of a predator-prey model}
\label{sec:predator}



We study the system
%
\begin{align*}
\dot{x} &= x (x-a) (1-x) - bxy  \\
\dot{y} &= xy - cy - d,
\end{align*}
%
with $a = 0.1$ and $b = 1.5$. The following could be an ecological interpretation of this system as a predator-prey model:

The $-d$ term represents a constant decline of species $y$; This would correspond to a linear decrease in $y$ over time, with slope $d$, if this was the only term present. Maybe an environmental agency eliminates a fixed number $d$ of $y$-type animals every time period, to keep the ecosystem balanced.

The $-cy$ term represents a proportional pressure on $y$, corresponding to an exponential decline of $y$ over time with time constant $1/c$ (i.e. faster decline for larger $c$). There might be a fixed amount of resources available for the $y$ species. Then, a larger number of $y$ animals will result in a proportionally smaller amount of resources per animal.

The $xy$ term represents a growth of $y$ that is both proportional to the other species and to itself. For constant $x$, this would correspond to exponential growth of $y$ with time constant $1/x$ (i.e. faster growth for more $x$). $y$ could be a multiplying parasite, and $x$ could be its host.

The $-bxy$ term represents a decline of the $x$ species proportional to both itself and to the other species $y$. For constant $y$, this would correspond to an exponential decline of $x$ with time constant $1/(by)$. The $y$ parasite might be pathological for $x$. Both more parasites $y$ and more hosts $x$ yield a higher probability of transmitting the parasite between hosts.

Finally, the $x (1-x)$ factor of the first term describes logistic growth (i.e. exponential growth from the origin, which switches halfway to exponential decay up to a carrying capacity -- which is $1$ in this case). This is a common model for constrained species growth. The $(x-a)$ multiplier has the effect that the growth does not start until $x$ reaches $a$: for $x < a$, the species will decline instead of grow. This could model the fact that more than a few individuals are necessary for succesful long-term reproduction.

In summary, the parameters $(a,b,c,d)$ could be given the following names:

\begin{deflist}
\notation{a}{Minimum prey population}
\notation{b}{Predator lethality}
\notation{c}{Predator decay rate}
\notation{d}{External predator elimination rate}
\end{deflist}




\section{A qualitative study for $d = 0$}
% 0,1  1,5     0,38  0,48

Simulating the system for $y = 0$ confirms the predictions made above for the standalone behaviour of $x(t)$ (\cref{fig:time_sim}): logistic growth above the threshold $a$, and decay to zero below this threshold.

\begin{figure}
\img[1]{time_sim}
\captionn{Prey without predators}{Simulated trajectories $x(t)$ for $y = 0$ and different initial values $x_0$ (from top to bottom: \numlist{1.5; 1.1; 1.0; 0.9; 0.5; 0.13; 0.1; 0.07; 0}). Note the stable fixed points at $0$ and at the carrying capacity $1$, and the unstable fixed point at $a = 0.1$.}
\label{fig:time_sim}
\end{figure}

The system has four fixed points for $d = 0$. Three of these lie on the $x$-axis. They are listed in \cref{tab:fpx}, together with the eigenvalues and corresponding eigenvectors of the Jacobian in these points. The fourth fixed points has coordinates $(c,\ (c-a)(1-c)/b)$, and the eigenstructure of its Jacobian is rather more.. complex. Its trace $\tau$ and determinant $\Delta$ have simpler analytical expressions however: $\tau = c(1+a-2c)$ and $\Delta = c(c-a)(1-c)$. The following paragraphs describe the topological structure near these four fixed points.

\begin{table}
\begin{tabular}{@{}llll@{}}
\toprule
Fixed point  & Eigenvalues  & Eigenvectors \\
\midrule
\multirow{2}{*}
{$(0,\ 0)$}  & $-a$         & $(1,\ 0)$             \\[0.2em]
             & $-c$         & $(0,\ 1)$             \\[1.2em]
\multirow{2}{*}
{$(a,\ 0)$}  & $a-a^2$      & $(1,\ 0)$             \\[0.2em]
             & $a-c$        & $(ab/(c-a^2),\ 1)$    \\[1.2em]
\multirow{2}{*}
{$(1,\ 0)$}  & $a-1$        & $(1,\ 0)$             \\[0.2em]
             & $1-c$        & $(-b/(2-c-a),\ 1)$    \\
\bottomrule
\end{tabular}
\captionn{Fixed points on the $\bm{x}$-axis}{$(x,y)$-coordinates of the fixed points, and eigenvalue-eigenvector pairs of the Jacobian.}
\label{tab:fpx}
\end{table}

$(0,\ 0)$ is an attractor node for $c > 0$. In the common case that $c > a = 0.1$, the $x$-axis is the slow eigendirection. When $c < a$, the $y$-axis is the slow eigendirection.

$(a,\ 0)$ is a saddle when $c > a$. Its stable manifold is then the $x$-axis, and its unstable manifold is locally spanned by $(ab/(c-a^2),\ 1)$, which is an upwards pointing vector, rotated slightly right. When $c < a$, both dimensions are unstable, and $(a,\ 0)$ is then a repellor node. For $c > a^2$, the slow eigendirection is $(ab/(c-a^2),\ 1)$. For $c < a^2$, the $x$-axis is the slow eigendirection.

$(1,\ 0)$ is a saddle when $c < 1$. With $a = 0.1$, the stable manifold is the $x$-axis, and the unstable manifold is locally spanned by $(-b/(2-c-a),\ 1)$, which is an upwards pointing vector, rotated slightly \emph{left}. When $c > 1$, both dimensions become stable, and $(1,\ 0)$ is then an attracting node. Because $c < 1.5$ in this exercise, the slow eigendirection is $(-b/(2-c-a),\ 1)$.

The final fixed point, $(c,\ (c-a)(1-c)/b)$, has a wider range of behaviours (\cref{fig:eigenheart}). It is a saddle for $c > 1$, and for $0 < c < a$. It is an unstable node for $a < c < c_u$, an unstable spiral for $c_u < c < (a+1)/2$, a stable spiral for $(a+1)/2 < c < c_s$, and a stable node for $c_s < c < 1$. $c_u$ and $c_s$ are the solutions to $4 \Delta = \tau^2$ (\citefull{Strogatz1994}). For $a = 0.1$, $c_u$ and $c_s$ are numerically determined to be $c_u = 0.1259..$ and $c_s = 0.8783..$.

\begin{figure}
\img[0.66]{eigenheart} \\[5em]
\img[0.88]{stability_diagram}
\captionn{Linear system analysis of the fixed point $\bm{(c,\  (a-c)(c-1)/b)}$}{\Top: eigenvalues of the Jacobian at the fixed point, for $a = 0.1$ and $c \in [0, 1.5]$ ($c = 0$ in the origin, and $c = 1.5$ at the far right for $\lambda_2$ and at the far left, outside the figure, for $\lambda_1$). \Bottom: trace $\tau$ and determinant $\Delta$ of the Jacobian at the fixed point, for $a = 0.1$ and $c \in [0, 1.04]$ (for larger $c$, the $(\tau, \Delta)$ curve simply extends further down in the 3rd quadrant). u.n.: unstable nodes.}
\label{fig:eigenheart}
\end{figure}

The separatrices between basins of attraction are the stable manifolds of saddles. The stable manifold of a saddle at $(x^*,\ y^*)$ can be found numerically by simulating a trajectory back in time. The initial point of this trajectory is $(x^* + \Delta x,\ y^* + \Delta y)$, where $\Delta x$ and $\Delta y$ are small perturbations on the line spanned by the eigenvector of the Jacobian corresponding to this stable manifold (i.e. the eigenvector with a negative eigenvalue).

This technique is used in \cref{fig:beautiful_phase_spaces} to draw the separatrices between basins of attraction. Note that the separatrices themselves are also one-dimensional `manifolds of attraction', namely for the saddles.

\begin{landscape}
\begin{figure}
\vspace{-5.2em}
\img[0.5]{phase_space_00000}
\img[0.5]{phase_space_04300}
\img[0.5]{phase_space_04460} \\[2.4em] % c_heterocline
\img[0.5]{phase_space_04700}
\img[0.5]{phase_space_05300}
\img[0.5]{phase_space_05500} \\[2.4em]
\img[0.5]{phase_space_07000}
\img[0.5]{phase_space_09200}
\img[0.5]{phase_space_11000} \\[0.3em]
\captionn{Phase space diagrams for different predator decay rates $\bm{c}$}{Colored circles represent fixed points. Their manifolds of attraction are indicated with the same color. (At $c = 0$, the entire $y$-axis consists of fixed points). The limit cycle for $c_h = 0.446... < c < 0.55$, its basin of attraction, and its period $T$ are indicated in purple.}
\label{fig:beautiful_phase_spaces}
\end{figure}
\end{landscape}

\Cref{fig:beautiful_phase_spaces} is consistent with the analysis made above: the origin is indeed always an attractor node for $c > 0$; the role of $(a,\ 0)$ indeed switches from repellor node to saddle (top row), just as $(1,\ 0)$ switches from saddle to attractor node (bottom row); and finally, the fixed point $(c,\ (c-a)(1-c)/b)$ takes on the subsequent roles predicted in \cref{fig:eigenheart}: from unstable spiral (top and middle row), to center (middle row, rightmost column), to stable spiral, attractor node, and saddle (bottom row).

This sequence of behaviours of the last fixed point can also be observed by following the eigenvalues $\lambda_i$ of its Jacobian in the complex plane (\cref{fig:eigenheart}). For $c > 1$, the eigenvalues start out on opposite ends of the real axis, and the fixed point is indeed a saddle. When $c$ then decreases, the eigenvalues move closer together, until the right eigenvalue (say $\lambda_2$) crosses the origin and enters stable territory, at $c = 1$. The fixed point is then a stable node, until $c = c_s = 0.8783..$. There, the eigenvalues meet on the negative real axis, only to immediately part again. The eigenvalues leave the real axis in opposite directions, tracing symmetric paths that bend towards the $y$-axis. The fixed point is then a stable spiral, up until the eigenvalues cross the $y$-axis. This happens at $c = 0.55$, and the fixed point is then locally surrounded by closed orbits. When $c$ decreases further, the eigenvalues continue their heart-shaped paths in the right half plane, while the fixed point is an unstable spiral. The eigenvalues meet again on the positive real axis, when $c = c_u = 0.1259..$. The fixed point is then a repellor node, while the eigenvalues move in opposite directions on the real axis, towards their initial locations (left for $\lambda_1$, right for $\lambda_2$). At $c = a = 0.1$, $\lambda_1$ crosses into the left half plane, making the fixed point a saddle point. This continues until $c = 0$, when both eigenvalues arrive at the origin, and the fixed point disappears.

Between $0.446... < c < 0.55$, the phase portrait contains a limit cycle (the purple orbit in \cref{fig:beautiful_phase_spaces}). The parameter value where the limit cycle first appears is determined numerically to be $c_h =0.44597197...$. For increasing $c$, the limit cycle tightens around the central fixed point, until it coalesces with it at $c = (1+a)/2 = 0.55$. The fixed point is then an attracting spiral with a large basin of attraction. Going back in the other direction, for decreasing $c$, both the orbit period and the limit cycle become increasingly large. In the limit for $c = c_h$, the limit cycle coincides with the stable manifolds of the two neighbouring saddle points ($(a,\ 0)$ and $(1,\ 0)$), and the orbit period becomes infinite. At this bifurcation point $c = c_h$, the two saddle points on the $x$-axis are connected in a heteroclinic orbit. Analogously to the `homoclinic bifurcation' from \citefull{Strogatz1994}, this bifurcation could be called a `heteroclinic bifurcation'.

\begin{figure}
\img[0.6]{xy_sim_harmonic}
\img[0.6]{xy_sim_heteroclinic}
\captionn{Limit cycle simulations}{Simulated trajectories $x(t)$ and $y(t)$ for a near-harmonic oscillation (left, $c = 0.55$) and a near-heteroclinic orbit (right, $c = 0.46$). In both simulations, the initial phase point was $(0.42, 0.12)$.}
\label{fig:lc_sims}
\end{figure}

These differences in limit cycle behaviour can be observed in \cref{fig:lc_sims}. A near-harmonic oscillation yields regular, sine-like time series. Their amplitude is roughly the same as their initial amplitude (closed orbit-like behaviour). On the other hand, a near-heteroclinic cycle yields more exotic time series, with a longer period. The amplitude also grows quickly to the limit cycle amplitude, irrespective of the initial condition. Finally, each oscillation period contains a long interval where both time series are minimal and near constant. This happens when the trajectory comes close to the slow, stable manifold on the $x$-axis
