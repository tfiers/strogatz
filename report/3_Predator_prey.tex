% !tex root = ./NLS_Report.tex
\graphicspath{{../figures/3/}}


\chapter{Study of a predator-prey model}
% 0,1  1,5     0,38  0,48


We study the system
%
\begin{align*}
\dot{x} &= x (x-a) (1-x) - bxy  \\
\dot{y} &= xy - cy - d,
\end{align*}
%
with $a = 0.1$ and $b = 1.5$. The following could be an ecological interpretation of this system as a predator-prey model:

The $-d$ term represents a constant decline of species $y$; This would correspond to a linear decrease in $y$ over time, with slope $d$, if this was the only term present. Maybe an environmental agency eliminates a fixed number $d$ of $y$-type animals every time period, to keep the ecosystem in check.

The $-cy$ term represents a proportional pressure on $y$, corresponding to an exponential decline of $y$ over time with time constant $1/c$ (i.e. faster decline for larger $c$). There might be a fixed amount of resources available for the $y$ species. Then, a larger number of $y$ animals will result in a proportionally smaller amount of resources per animal.

The $xy$ term represents a growth of $y$ that is both proportional to the other species and to itself. For constant $x$, this would correspond to exponential growth of $y$ with time constant $1/x$ (i.e. faster growth for more $x$). $y$ could be a multiplying parasite, and $x$ could be its host.

The $-bxy$ term represents a decline of the $x$ species proportional to both itself and to the other species $y$. For constant $y$, this would correspond to an exponential decline of $x$ with time constant $1/(by)$. The $y$ parasite might be pathological for $x$. Both more parasites $y$ and more hosts $x$ yield a higher probability of transmitting the parasite between hosts.

Finally, the $x (1-x)$ factors of the first term describe logistic growth (i.e. exponential growth, which switches to exponential slowing down when the carrying capacity of $1$ is nearly reached). This is a common model for constrained species growth. The $(x-a)$ multiplier has the effect that the growth does not start until $x$ reaches $a$: for $x < a$, the species will decline instead of grow. This could model the fact that more than a few individuals are necessary for succesful long-term reproduction.



\section{A qualitative study for $d = 0$}


