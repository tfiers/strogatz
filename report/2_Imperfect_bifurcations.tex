% !tex root = ./NLS_Report.tex


\chapter{Imperfect bifurcations}

When $h$ is varied, we observe saddle-node bifurcations for $r > 0$ (see \cref{fig:h_u}). For $r = 0$, these two points coalesce into a single degenerate bifurcation point at $h = 0$, where $u$ experiences critical slowing down as it approaches the origin.

\begin{figure}
\img[0.4]{r_-1}
\img[0.4]{r_0}
\img[0.4]{r_1}
\captionn{$\bm{(h,u)}$-bifurcation diagrams}{Blue dots are stable equilibria, red dots are unstable equilibria, and large red circles indicate saddle-node bifurcations. Left: $r = -1$. (Phase space topologies for other $r < 0$ are equivalent). Center: $r = 0$. Right: $r = 1$. (Phase space topologies for other $r > 0$ are equivalent).}
\label{fig:h_u}
\end{figure}

% When on the other hand 

\begin{figure}
\img[0.4]{h_neg}
\img[0.4]{h_0}
\img[0.4]{h_pos}
\captionn{$\bm{(r,u)}$-bifurcation diagrams}{Colors as in \cref{fig:h_u}. Left: $h = -0.1$. Center: $h = 0$. Right: $h = 0.1$.}
\label{fig:r_u}
\end{figure}

\begin{figure}
\img[0.4]{bf_branch_u_r}
\img[0.4]{bf_branch_u_h}
\img[0.4]{bf_branch_r_h}
\captionn{The fold curve}{Continuation of the saddle-node bifurcations (the large red dots in \cref{fig:h_u,fig:r_u}), projected on the $(u,r)$, $(u,h)$, and $(r,h)$ planes.}
\label{fig:fold_curve}
\end{figure}
